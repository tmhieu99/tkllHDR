\documentclass[12pt, a4paper, oneside]{article}

\setlength{\parskip}{1ex plus 0.5ex minus 0.2ex}
\renewcommand{\baselinestretch}{1.1}

\usepackage[T1]{fontenc}
\usepackage{libertine}
\usepackage[english]{babel}
\usepackage[protrusion=true,expansion=true]{microtype}	
\usepackage{amsmath,amsfonts}                               % Math packages
\usepackage[pdftex]{graphicx}	
\usepackage{url}
\usepackage{verbatim} 
\usepackage{hyperref}
\hypersetup{colorlinks = true, allcolors=black}
\usepackage{cite}
\usepackage[utf8]{inputenc}
\usepackage[section]{placeins}
\usepackage{mathtools}
\usepackage{array}
\usepackage[colorinlistoftodos]{todonotes}
\usepackage[T1]{fontenc}
\usepackage{bookmark}
\usepackage{caption}
\usepackage{float}
\usepackage{enumitem}
\usepackage{textcomp}
\usepackage[sort,nameinlink]{cleveref}
\usepackage{amssymb}
\usepackage{xcolor}
\usepackage{booktabs}
\usepackage{multirow}
\usepackage{listings}
\usepackage{subcaption}
\usepackage{adjustbox}
\usepackage{multirow}
\usepackage{gensymb}
\usepackage{algorithm}
\usepackage{algpseudocode}
\usepackage{lipsum}
\usepackage{geometry}
\usepackage{sectsty}                                        % Custom sectioning
\usepackage{fancyhdr}                                       % Custom headers/footers

\geometry{
	a4paper,
	total={170mm,257mm},
	left=25mm,
	right=20mm,
	top=30mm,
	bottom=30mm
}

%%% Custom headers/footers (fancyhdr package)
\pagestyle{fancyplain}
\fancyhead{}									    		% No page header
\fancyfoot[L]{}									    		% Empty 
\fancyfoot[C]{}									    		% Empty
\fancyfoot[R]{\thepage}							    		% Pagenumbering
\renewcommand{\headrulewidth}{0pt}                          % Remove header underlines
\renewcommand{\footrulewidth}{0pt}                          % Remove footer underlines
\setlength{\headheight}{13.6pt}
\lstset{breaklines}
\definecolor{MyDarkGreen}{rgb}{0.0,0.4,0.0}
\lstloadlanguages{C}
\lstset{ % Use Perl in this example
	frame=single, % Single frame around code
	basicstyle=\small\ttfamily, % Use small true type font underlined and blue
	identifierstyle=, % Nothing special about identifiers                                         
	showstringspaces=false, % Don't put marks in string spaces
	tabsize=3, % 5 spaces per tab
	%
	% Put standard Perl functions not included in the default language here
	morekeywords={rand},
	%
	% Put Perl function parameters here
	morekeywords=[2]{on, off, interp},
	%
	% Put user defined functions here
	%	morekeywords=[3]{test},
	%
	numbers=left, % Line numbers on left
	firstnumber=1, % Line numbers start with line 1
	stepnumber=5 % Line numbers go in steps of 5
}

%%% Maketitle metadata
\newcommand{\horrule}[1]{\rule{\linewidth}{#1}} 	% Horizontal rule

\lstset{breaklines}

\lstloadlanguages{C}

\lstset{ % Use Perl in this example
	frame=single, % Single frame around code
	basicstyle=\small\ttfamily, % Use small true type font underlined and blue
	identifierstyle=, % Nothing special about identifiers                                         
	showstringspaces=false, % Don't put marks in string spaces
	tabsize=3, % 5 spaces per tab
	%
	% Put standard Perl functions not included in the default language here
	morekeywords={rand},
	%
	% Put Perl function parameters here
	morekeywords=[2]{on, off, interp},
	%
	% Put user defined functions here
	%	morekeywords=[3]{test},
	%
	numbers=left, % Line numbers on left
	firstnumber=1, % Line numbers start with line 1
	stepnumber=1 % Line numbers go in steps of 5
}

%%% Begin document
\begin{document}
    \thispagestyle{empty}
    \begin{titlepage}
        \begin{center}
    	\normalfont \normalsize \textsc{Faculty of Computer Science and Engineering\\Ho Chi Minh City University of Technology}\\ [48pt]
    	{\large -- Computer Architecture -- Semester 182 -- }
        \horrule{0.5pt} \\[0.4cm]
    	\huge VERILOG-BASED SINGLE-CYCLE \\
    	\huge MIPS PROCESSOR
    	\horrule{2pt} \\[0.5cm]
    	
        \includegraphics[width=6cm]{BK.png} \\
        \vspace{1cm}
    	
        \scalebox{0.6}{%
            \vspace{3cm}
        	\begin{tabular}{|l|l|c|}                                                           \hline
            \textbf{No.}    & \multicolumn{1}{c|}{\textbf{Name}}    & \textbf{Student's ID} \\ \hline
            \textbf{1}      & Nguyen Minh Dang                      & 1752170               \\ \hline
            \textbf{3}      & Le Nguyen An Khuong                   & 1752305               \\ \hline
            \textbf{4}      & Nhan Ngoc Thien                       &  1752508               \\ \hline
            \end{tabular}
        }
        
        {\large Instructor: Dr. Pham Quoc Cuong}
        \end{center}
    \end{titlepage}
    \newpage
    \pagenumbering{roman}
    \tableofcontents
    \newpage
    \listoffigures
    \newpage
    \pagenumbering{arabic}
    \setcounter{page}{1}
    
    \suppressfloats %No figures on first page

%--------------------------------------------------------CONTENT--------------------------------------------------------%

    \section{Introduction}
 \hspace{0.5cm} The report is about designing a MIPS Single Clock Cycle using Verilog HDL. The report involves using De2i board to simulate the result. The MIPS Single Clock Cycle processor is constructed by implementing 32-bit 32 registers, and 10 memory locations- each is 8 bit wide - (due to limitation of resources). The result is presented on De2i board by 3 methods: 26 leds, 7-segment leds and LCD.
    \section{Design}
    \subsection{Datapath} 
    \begin{figure}[H]
    	\includegraphics[width=17.5cm]{datapath.png}
    	\caption{Single Cycle Full Datapath}
    	\label{fig:1}
    \end{figure}
    \subsection{Block Description}
    \subsubsection{Module SYS Master}
    $\bullet$ Top-Level Entity module: connect all other modules: Instruction Memories, Register Files, ALU, Data Memory, Control Unit,...\\
    \indent \indent- Display 26 bits low of output on leds, display 32 bits of output on 7-segment leds and display 32 bits output, PC, and 8 bits of SYS$\_$output$\_$sel on LCD.\\
    \indent \indent- SYS$\_$clk works as clock triggered to execute instructions.\\
    \indent \indent- SYS$\_$load is used to load 8 bits low of PC to demo the wanted instructions.\\
    \indent \indent- SYS$\_$output$\_$sel is describe as following:\\
    \indent \indent \indent \indent+ 0: Instructions Memory output is displayed.\\
    \indent \indent \indent \indent+ 1: output-1 of Register Files is displayed.\\
    \indent \indent \indent \indent+ 2: ALU result is displayed.\\
    \indent \indent \indent \indent+ 3: ALU status is displayed.\\
    \indent \indent \indent \indent+ 4: Data Memory result is displayed.\\
    \indent \indent \indent \indent+ 5: Control Unit output is displayed.\\    
    \indent \indent \indent \indent+ 6: ALU Control output is displayed.\\
    \indent \indent \indent \indent+ 7: PC address is displayed.\\
    \indent \indent \indent \indent+ 8: output-2 of Register Files is displayed.\\
    \indent \indent \indent \indent+ 9: Exception Handle output is displayed.
    
    \subsubsection{Instruction Memory}
    $\bullet$ Store the machine code of the program. \\
    $\bullet$ Receive the address of the current instruction.\\
    $\bullet$ Transfer the instruction to other block for execution.
    
    \subsubsection{Module Register Files}
    $\bullet$ Receive the address of the instruction and decode the instruction of the register (there are 32 32-bit registers).  Output is the value of the register. \\
    $\bullet$ If there is write signal, the input value will be written into the destination register. \\
    $\bullet$ If the write register is $\$$0 then the error signal will be active and send signal to Exception Handle module.
    
    \subsubsection{Module ALU}
    $\bullet$ Execute calculations(add, addi, sub, and, or, slt, branch, beq, jump,...). \\
    $\bullet$ Output is the result of the calculation and status signal if there are any exception (divided by 0, overflow,...)\\
    $\bullet$ ALU status is displayed as followed:\\
    \begin{center}
    	\begin{tabular}[H]{|l|l|l|l|}
    	 \hline
    	 Name & Bit & Description & Exception \\ \hline
    	 Zero & 7 & Active when ALU result equals zero &  \\ \hline
    	 Overflow & 6 & Active when ALU result overflows & x \\ \hline
    	 Carry &  5 & Active when ALU result has carry bit & \\ \hline
    	 Negative & 4 & Active when ALU result is negative & \\ \hline
    	 Invalid Address & 3 & Active when ALU result is not aligned & x \\ \hline
    	 Undefined & [2:0] & No description & \\ \hline
    	 
    	\end{tabular}
	\end{center}

    \subsubsection{Module Data Memory}
    $\bullet$ Create memory space :10 memory locations, each is 8 bit due to the resources limitation.\\
    $\bullet$ Access and write into the input location of the memory. 
    
    \subsubsection{Control Unit}
    $\bullet$ Receive the Operation bit from Instruction Memory to transfer the signal for other blocks to execute the given instruction.\\
    $\bullet$ The output of Control Unit is described as followed: \\
    \begin{center}
    \begin{tabular}{|l|l|}
    	\hline
    	Name & Bit \\ \hline
    	Jump & 10 \\ \hline
    	Branch & 9 \\ \hline
    	MemRead & 8 \\  \hline
    	MemWrite & 7 \\  \hline
    	Mem2Reg & 6 \\  \hline
    	ALUOp & [5:4] \\  \hline
    	Exception & 3 \\  \hline
    	ALUsrc & 2 \\  \hline
    	RegWrite & 1 \\  \hline
    	RegDst & 0 \\ \hline
    \end{tabular}
   \end{center} 
    
    \subsubsection{Module PC}
    $\bullet$ Giving the address of the current instruction \\
    $\bullet$ Receive the address of the next instruction to execute. \\
    $\bullet$ If exception occurs, PC won't be directed to next instruction.
    
     \subsubsection{Module ALU Control}
    $\bullet$ Receive function from Instruction Memory and ALU OpCode from the Control Unit.\\
    $\bullet$ Giving ALU block the compatible signal to perform the equivalent arithmetic operation. 
    
    \subsubsection{Module Exception Handle}
    \begin{tabular}{ll}
    	$\bullet$ Receive 8 bit of ALU status, 4 bit: Exception, MemRead, MemWrite, MemRead, Mem2Reg & \\ from Control Unit, and 1 bit from Register Files.  & \\
    	$\bullet$ Exceptions need to be handled:& \\ 
    	\quad \quad- Instruction syntax error. &\\
    	\quad \quad- Hardware error (write on register \$zero, unaligned). &\\
    	\quad \quad- Arithmetic exception (overflow). &\\
    	$\bullet$ The system will enable  block the Read and Write signal to prevent output error.&
    \end{tabular}

	\subsubsection{Module Mux}
	$\bullet$ Selection module with two inputs.
	\subsubsection{Module Sign Extend}
	$\bullet$ Input is an 16-bit integer, 5 bits shamt, output is an 32-bit extended integer number.
    
	\subsubsection{LCD}
	\begin{figure}[H]
		\includegraphics[width=16cm]{Moore_FSM.png}
		\caption{LCD Finite State Machine}
		\label{fig:LCD}
	\end{figure}
	$\bullet$ Manipulate the LCD to display necessary information of the system. \\
	$\bullet$ The LCD works by receiving and processing instructions from user. This module automates the process of sending the instructions into the LCD. \\
	$\bullet$ The figure above demonstrate the LCD's Finite State Machine \\
	\indent	$\bullet$ \textbf{INIT:} In this state the module initialized setup instructions to push into the LCD. \\
	\indent	$\bullet$ \textbf{PUSH:} In this state the module push instruction into the LCD (one instruction per time) \\ 
	\indent	$\bullet$ \textbf{LOAD:} In this state the module load the next instruction to be pushed. \\
	\indent	$\bullet$ \textbf{IDLE:} After pushing all the instructions, the module goes to IDLE state, which it will do nothing and wait for next set of instructions. \\
	
	Finally, all state can transit to INIT state with an Asynchronous reset signal.
	
	
\section{Implementation}
     \subsection{Prototype}
   \subsubsection{Module SYS-Master}
     \begin{lstlisting}
module SYS_Master(
  output [26:0] SYS_leds,
  output [0:6] HEX0, HEX1, HEX2, HEX3, HEX4, HEX5, HEX6, HEX7,
  output [7:0]	LCD_DATA,
  output		LCD_RW,LCD_EN,LCD_RS,
  input SYS_clk,
  input SYS_reset,
  input SYS_load,
  input [7:0] SYS_pc_load,
  input [7:0] SYS_output_sel,
  input CLOCK_50, SW
);
     \end{lstlisting}
     
     \subsubsection{Module Instructions Memory}
     \begin{lstlisting}
module inst_mem(
  output [31:0] imem_instruction,
  input  reset,
  input  [31:0] imem_pc
);
     \end{lstlisting}
     
      \subsubsection{Module Register Files}
     \begin{lstlisting}
module register_files(
  output [31:0] read_data_1,
  output [31:0] read_data_2,
  output error_toggle,
  input [4:0]   read_register_1, 
  input [4:0]   read_register_2, 
  input [4:0]   write_register, 
  input         write_switch,
  input [31:0] write_data,
  input clk, reset, enable
);
     \end{lstlisting}
     
      \subsubsection{Module ALU}
     \begin{lstlisting}
module alu(
  output [31:0] alu_result,
  output [7:0] alu_status,
  input signed [31:0] alu_operand_1, alu_operand_2,
  input  [3:0] alu_control
);
     \end{lstlisting}
     
     \subsubsection{Module Data Memory}
     \begin{lstlisting}
module data_memory(
  output [31:0] data_out,
  input[31:0] address,
  input[31:0] data_in,
  input mem_read, mem_write, clk, reset
);
     \end{lstlisting}
     
     \subsubsection{Module Control Unit}
     \begin{lstlisting}
module control_unit(
  output [10:0] control_signal,
  input [5:0] opcode
);
     \end{lstlisting}

     \subsubsection{LCD}
     \begin{lstlisting}
module lcd(
	input CLOCK,
	input RST,
	input CLK,
	input LOAD,
	input SW,
	input [31:0] DATA,
	input [31:0] PC,
	input [7:0]  SEL,
	output reg LCD_EN,
	output reg LCD_RS,
	output LCD_RW,
	output reg [7:0] LCD_DATA
);
     \end{lstlisting}
    \section{Result and Simulation}
    \section{Conclusions}
    \hspace{0.5cm}The project provide the opportunity to comprehend the function of a normal MIPS instruction, study deep into how the computer when each line of code is compiled; a long with that is the chance of practicing with Verilog. This project gives us the base knowledge for future programming as well as much greater project.
\end{document}
